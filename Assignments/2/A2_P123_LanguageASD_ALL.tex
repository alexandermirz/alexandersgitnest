\documentclass[]{article}
\usepackage{lmodern}
\usepackage{amssymb,amsmath}
\usepackage{ifxetex,ifluatex}
\usepackage{fixltx2e} % provides \textsubscript
\ifnum 0\ifxetex 1\fi\ifluatex 1\fi=0 % if pdftex
  \usepackage[T1]{fontenc}
  \usepackage[utf8]{inputenc}
\else % if luatex or xelatex
  \ifxetex
    \usepackage{mathspec}
  \else
    \usepackage{fontspec}
  \fi
  \defaultfontfeatures{Ligatures=TeX,Scale=MatchLowercase}
\fi
% use upquote if available, for straight quotes in verbatim environments
\IfFileExists{upquote.sty}{\usepackage{upquote}}{}
% use microtype if available
\IfFileExists{microtype.sty}{%
\usepackage{microtype}
\UseMicrotypeSet[protrusion]{basicmath} % disable protrusion for tt fonts
}{}
\usepackage[margin=1in]{geometry}
\usepackage{hyperref}
\hypersetup{unicode=true,
            pdftitle={Assignment 2 - Language Development in ASD - Part 1 - Explaining development},
            pdfauthor={Alexander Jakub},
            pdfborder={0 0 0},
            breaklinks=true}
\urlstyle{same}  % don't use monospace font for urls
\usepackage{graphicx,grffile}
\makeatletter
\def\maxwidth{\ifdim\Gin@nat@width>\linewidth\linewidth\else\Gin@nat@width\fi}
\def\maxheight{\ifdim\Gin@nat@height>\textheight\textheight\else\Gin@nat@height\fi}
\makeatother
% Scale images if necessary, so that they will not overflow the page
% margins by default, and it is still possible to overwrite the defaults
% using explicit options in \includegraphics[width, height, ...]{}
\setkeys{Gin}{width=\maxwidth,height=\maxheight,keepaspectratio}
\IfFileExists{parskip.sty}{%
\usepackage{parskip}
}{% else
\setlength{\parindent}{0pt}
\setlength{\parskip}{6pt plus 2pt minus 1pt}
}
\setlength{\emergencystretch}{3em}  % prevent overfull lines
\providecommand{\tightlist}{%
  \setlength{\itemsep}{0pt}\setlength{\parskip}{0pt}}
\setcounter{secnumdepth}{0}
% Redefines (sub)paragraphs to behave more like sections
\ifx\paragraph\undefined\else
\let\oldparagraph\paragraph
\renewcommand{\paragraph}[1]{\oldparagraph{#1}\mbox{}}
\fi
\ifx\subparagraph\undefined\else
\let\oldsubparagraph\subparagraph
\renewcommand{\subparagraph}[1]{\oldsubparagraph{#1}\mbox{}}
\fi

%%% Use protect on footnotes to avoid problems with footnotes in titles
\let\rmarkdownfootnote\footnote%
\def\footnote{\protect\rmarkdownfootnote}

%%% Change title format to be more compact
\usepackage{titling}

% Create subtitle command for use in maketitle
\providecommand{\subtitle}[1]{
  \posttitle{
    \begin{center}\large#1\end{center}
    }
}

\setlength{\droptitle}{-2em}

  \title{Assignment 2 - Language Development in ASD - Part 1 - Explaining
development}
    \pretitle{\vspace{\droptitle}\centering\huge}
  \posttitle{\par}
    \author{Alexander Jakub}
    \preauthor{\centering\large\emph}
  \postauthor{\par}
      \predate{\centering\large\emph}
  \postdate{\par}
    \date{{[}DATE{]}}


\begin{document}
\maketitle

\section{Assignment 2}\label{assignment-2}

In this assignment you will have to discuss a few important questions
(given the data you have). More details below. The assignment submitted
to the teachers consists of: - a report answering and discussing the
questions (so we can assess your conceptual understanding and ability to
explain and critically reflect) - a link to a git repository with all
the code (so we can assess your code)

Part 1 - Basic description of language development - Describe your
sample (n, age, gender, clinical and cognitive features of the two
groups) and critically assess whether the groups (ASD and TD) are
balanced - Describe linguistic development (in terms of MLU over time)
in TD and ASD children (as a function of group). - Describe how parental
use of language (in terms of MLU) changes over time. What do you think
is going on? - Include individual differences in your model of language
development (in children). Identify the best model.

Part 2 - Model comparison - Discuss the differences in performance of
your model in training and testing data - Which individual differences
should be included in a model that maximizes your ability to
explain/predict new data? - Predict a new kid's performance (Bernie) and
discuss it against expected performance of the two groups

Part 3 - Simulations to plan a new study - Report and discuss a power
analyses identifying how many new kids you would need to replicate the
results

The following involves only Part 1.

\subsection{Learning objectives}\label{learning-objectives}

\begin{itemize}
\tightlist
\item
  Summarize and report data and models
\item
  Critically apply mixed effects (or multilevel) models
\item
  Explore the issues involved in feature selection
\end{itemize}

\section{Quick recap}\label{quick-recap}

Autism Spectrum Disorder is often related to language impairment.
However, this phenomenon has not been empirically traced in detail: i)
relying on actual naturalistic language production, ii) over extended
periods of time.

We therefore videotaped circa 30 kids with ASD and circa 30 comparison
kids (matched by linguistic performance at visit 1) for ca. 30 minutes
of naturalistic interactions with a parent. We repeated the data
collection 6 times per kid, with 4 months between each visit. We
transcribed the data and counted: i) the amount of words that each kid
uses in each video. Same for the parent. ii) the amount of unique words
that each kid uses in each video. Same for the parent. iii) the amount
of morphemes per utterance (Mean Length of Utterance) displayed by each
child in each video. Same for the parent.

This data is in the file you prepared in the previous class.

NB. A few children have been excluded from your datasets. We will be
using them next week to evaluate how good your models are in assessing
the linguistic development in new participants.

This RMarkdown file includes 1) questions (see above). Questions have to
be answered/discussed in a separate document that you have to directly
send to the teachers. 2) A break down of the questions into a guided
template full of hints for writing the code to solve the exercises. Fill
in the code and the paragraphs as required. Then report your results in
the doc for the teachers.

REMEMBER that you will have to have a github repository for the code and
send the answers to Kenneth and Riccardo without code (but a link to
your github/gitlab repository). This way we can check your code, but you
are also forced to figure out how to report your analyses :-)

Before we get going, here is a reminder of the issues you will have to
discuss in your report:

1- Describe your sample (n, age, gender, clinical and cognitive features
of the two groups) and critically assess whether the groups (ASD and TD)
are balanced 2- Describe linguistic development (in terms of MLU over
time) in TD and ASD children (as a function of group). 3- Describe how
parental use of language (in terms of MLU) changes over time. What do
you think is going on? 4- Include individual differences in your model
of language development (in children). Identify the best model.

\section{Let's go}\label{lets-go}

\subsubsection{Loading the relevant
libraries}\label{loading-the-relevant-libraries}

Load necessary libraries : what will you need? - e.g.~something to deal
with the data - e.g.~mixed effects models - e.g.~something to plot with

\subsubsection{Define your working directory and load the
data}\label{define-your-working-directory-and-load-the-data}

If you created a project for this class and opened this Rmd file from
within that project, your working directory is your project directory.

If you opened this Rmd file outside of a project, you will need some
code to find the data: - Create a new variable called locpath
(localpath) - Set it to be equal to your working directory - Move to
that directory (setwd(locpath)) - Load the data you saved last time (use
read\_csv(fileName))

\subsubsection{Characterize the participants (Exercise
1)}\label{characterize-the-participants-exercise-1}

\subsection{Let's test hypothesis 1: Children with ASD display a
language impairment (Exercise
2)}\label{lets-test-hypothesis-1-children-with-asd-display-a-language-impairment-exercise-2}

\subsubsection{Hypothesis: The child's MLU changes: i) over time, ii)
according to
diagnosis}\label{hypothesis-the-childs-mlu-changes-i-over-time-ii-according-to-diagnosis}

How would you evaluate whether the model is a good model?

\subsection{Let's test hypothesis 2: Parents speak equally to children
with ASD and TD (Exercise
3)}\label{lets-test-hypothesis-2-parents-speak-equally-to-children-with-asd-and-td-exercise-3}

\subsubsection{Adding new variables (Exercise
4)}\label{adding-new-variables-exercise-4}

\subsection{Part 2}\label{part-2}

\subsubsection{Exercise 1) Testing model
performance}\label{exercise-1-testing-model-performance}

\subsubsection{Exercise 2) Model Selection via Cross-validation (N.B:
ChildMLU!)}\label{exercise-2-model-selection-via-cross-validation-n.b-childmlu}

\subsubsection{Exercise 3) Assessing the single
child}\label{exercise-3-assessing-the-single-child}


\end{document}
